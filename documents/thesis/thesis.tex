% Preamble
% ---
\documentclass[a4paper]{report}

% Packages
% ---
\usepackage{float}
\restylefloat{table}
\usepackage[font=small,labelfont=bf,center]{caption}
\usepackage{amsfonts} 
\usepackage{amsmath} % Advanced math typesetting
\usepackage[utf8]{inputenc} % Unicode support (Umlauts etc.)
\usepackage[english]{babel} % Change hyphenation rules
\usepackage{hyperref} % Add a link to your document
\usepackage{graphicx} % Add pictures to your document
\usepackage{listings} % Source code formatting and highlighting
\usepackage{multicol}
\usepackage[svgnames]{xcolor}
\usepackage{listings}
\usepackage{slashbox}
\usepackage{wrapfig}
\usepackage{multirow}
\usepackage[
backend=bibtex,
style=authoryear-comp,
]{biblatex}
\addbibresource{research.bib}

\usepackage[numbib,nottoc]{tocbibind}


\begin{document} 
	
	\author{Christian Ames\"oder}

\title {RNNs to analyse causality in timeseries for jSDMs}

\begin{titlepage}
	\centering
	\includegraphics[width=0.15\textwidth]{ur-logo-bildmarke-grau.jpg}\par\vspace{1cm}
	{\scshape \LARGE University of Regensburg\par}
	\vspace{1cm}
	{\scshape\Large Master Thesis\par}
	\vspace{1.5cm}
	{\huge\bfseries RNNs to analyse causality in time series for jSDMs\par}
	\vspace{2cm}
	{\Large\itshape Christian Ames\"oder\par}
	\vfill
	supervised by\par
	Maximilian Pichler and Prof. Dr.~Florian \textsc{Hartig}
	
	\vfill
	
	second corrector \par
	Prof. Dr.~Elmar \textsc{Lang}
	\vfill
	
	% Bottom of the page
	{\large \today\par}
\end{titlepage}

	\tableofcontents

	\chapter{Introduction}
	Recently the interest in causality in all fields of science has risen. For static data a conclusion whether correlation equals causation is not possible. Therefore a look on time series data is inevitable. In time series analysis plenty of methods have been developed to imply causality from this data. Starting in econometrics, where \cite{Granger.1969} first introduces the so called Granger Causality it is now used in a wide array of fields, such as in econometrics to analyse the stock markets,in neuroscience where it is used in the analysis of electroencephalograms (EEG) as well as in climate research. Similar research is also done in ecology, to understand the underlying structure of an ecological community. The rapid cost decrease of next-generation sequencing lead to environmental DNA being a valid method for generate huge time series data about an ecosystems and its species. 
	Originating from species distribution models (SDMs, ) which model an ecosystem based on spatial environmental variables, joint SDMs (jSDMs) also consider species-species interaction in the modelling process. 
	Approaches to correctly model jSDMs include the use of convergent cross maps (CCM, \cite{Sugihara.2012})and Granger Causality methods. However, it seems that CCMs don't provide any significant advantage compared to a simple multivariate autoregressive model for the sake of analysing causal relations (\cite{Barraquand.2021}). The path of using Granger causality to determine causation in time series data seems to be the most promising. Granger causality can be determined in many ways. While traditional methods for jSDMS rely on training GLMMs, recent approaches also suggest the use of Neural Networks (\cite{Montalto.2015}, \cite{Tank.2021}). This seems logical, since Granger causality determination heavily relies on how good one variable can be predicted with another.
	
	
	\chapter{Methods}
	
	\section{Time Series}
	\textbf{Vector autoregressive (VAR)}\\
	For time series with T instances \textbf{x} $ = (x_t)_{t\in [1:T]}$ VAR is defined as following:
	
	
	\begin{equation} \label {lasso}
		x_t = c+ A_1x_{t-1}+...+ A_p x_{t-p}+\epsilon_t
	\end{equation}

	
	
	\section{Granger Causality approach}
	Granger causality for two time series x and y is defined whether past x values improve a prediction of y or not. (\cite{Granger.1969}). 
	There have been various approaches to this topic. Approaches relevant to this research project are briefly introduced here.
	
	\subsection{Multivariate autoregressive model (MAR)} 
	A MAR(p), for p $\geq$ 1, model is defined as comparing $\eta_t$ and $\epsilon_t$ in the following two autoregressive models:
	\begin{align}
		y_t =& \sum_{i=1}^{p} \alpha_i y_{t-1}+\eta_t, \; \eta_t \sim \mathcal{N}(0,\,\sigma_\eta^{2})\\
		y_t = &\sum_{i=1}^{p} \alpha_i y_{t-1}+ \sum_{i=1}^{p}\beta_{2i}y_{t-i}+\epsilon_t, \; \epsilon_t \sim \mathcal{N}(0,\,\sigma_\epsilon^{2})
	\end{align}
	If $\sigma_\epsilon^{2} \; < \; \sigma_\eta^{2} $ and the difference is statistically significant you conclude, x Granger causes y (\cite{Geweke.1982}).\\
	
	\subsection{Neural Granger}
	There also exist other approaches using Neural Networks such as in \cite{Tank.2020}, where on each time series a predictive neural network is trained and afterwards all input layer weights are jointly generalized using lasso regression. A Granger causal effect is determined here whether all input weights of a specific time series are set to 0 or not, where 0 implies no causal effect. 
	
	
	
	%\chapter{Results}
	%\chapter{Discussion}	
	%\chapter{Conclusion}
	\chapter{Appendix}
	
	\printbibliography
\end{document}
