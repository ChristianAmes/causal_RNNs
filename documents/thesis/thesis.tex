% Preamble
% ---
\documentclass[a4paper]{report}

% Packages
% ---
\usepackage{float}
\restylefloat{table}
\usepackage[font=small,labelfont=bf,center]{caption}
\usepackage{amsfonts} 
\usepackage{amsmath} % Advanced math typesetting
\usepackage[utf8]{inputenc} % Unicode support (Umlauts etc.)
\usepackage[english]{babel} % Change hyphenation rules
\usepackage{hyperref} % Add a link to your document
\usepackage{graphicx} % Add pictures to your document
\usepackage{listings} % Source code formatting and highlighting
\usepackage{multicol}
\usepackage[svgnames]{xcolor}
\usepackage{listings}
\usepackage{slashbox}
\usepackage{wrapfig}
\usepackage{multirow}
\usepackage[
backend=bibtex,
style=authoryear-comp,
]{biblatex}
\addbibresource{research.bib}

\usepackage[numbib,nottoc]{tocbibind}


\begin{document} 
	
	\author{Christian Ames\"oder}

\title {RNNs to analyse causality in timeseries for jSDMs}

\begin{titlepage}
	\centering
	\includegraphics[width=0.15\textwidth]{ur-logo-bildmarke-grau.jpg}\par\vspace{1cm}
	{\scshape \LARGE University of Regensburg\par}
	\vspace{1cm}
	{\scshape\Large Master Thesis\par}
	\vspace{1.5cm}
	{\huge\bfseries RNNs to analyse causality in time series for jSDMs\par}
	\vspace{2cm}
	{\Large\itshape Christian Ames\"oder\par}
	\vfill
	supervised by\par
	Maximilian Pichler and Prof. Dr.~Florian \textsc{Hartig}
	
	\vfill
	
	second corrector \par
	Prof. Dr.~Elmar \textsc{Lang}
	\vfill
	
	% Bottom of the page
	{\large \today\par}
\end{titlepage}


	\tableofcontents


	\chapter{Introduction}
	
	\chapter{Methods}
	
	\chapter{Results}
	\section{Time Series}
	In this work I define multivariate time series as n variables $X_1,..., X_n$ with k instances each. This results in a matrix $A \in M^(n \times k))$ where row i corresponds to variable $X_i$. In ecology each X represents one taxa included in the study. Since often times environmental variables are also known we can expand our time series with an additional data matrix $B \in M^(l \times n)$ where l is the number of environmental variables available.
	The goal is to acquire the causality matrix C which, similar to a covariance matrix, contains the causal inferences between all variables.
	
	
	\section{Granger Causality}
	In time series analysis you say, that variable X Granger causes variable Y if information about past X values improves a prediction of Y \cite{Granger.1969}. 
	
	\chapter{Discussion}
	
	\chapter{Conclusion}
	
	
	\printbibliography
\end{document}
